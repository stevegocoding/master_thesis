\documentclass[12pt]{report}
\usepackage{setspace}
\usepackage{here}
\usepackage{cuthesis}
\usepackage{amsmath}
\usepackage[font=small,format=plain,labelfont=bf,up,textfont=up]{caption}
\usepackage[T1]{fontenc}
\usepackage{graphicx}
\usepackage{subfigure}
\usepackage{float}
\usepackage{listings}
\usepackage{courier}
\usepackage[ruled,vlined]{algorithm2e}
\usepackage{verbatim}
\usepackage[toc, page]{appendix}
\usepackage[numbers]{natbib}
\usepackage{pbox}
\usepackage{comment}

\author{Guangfu Shi}
\title{Efficient Rendering with Dynamic Lighting Using Photons Queue and Incremental Update Algorithm}
\degree{Master of Computer Science}
\dept{Computer Science}       						% default is Comp.Sci.


% Source Code Formatting
\lstset{
         %basicstyle=\footnotesize\ttfamily, % Standardschrift
         basicstyle=\small\ttfamily,
         %numbers=left,               % Ort der Zeilennummern
         numberstyle=\tiny,          % Stil der Zeilennummern
         %stepnumber=2,               % Abstand zwischen den Zeilennummern
         numbersep=5pt,              % Abstand der Nummern zum Text
         tabsize=2,                  % Groesse von Tabs	
         extendedchars=true,         %
         breaklines=true,            % Zeilen werden Umgebrochen
         %keywordstyle=\color{red},
         %frame=t|b,
 %        keywordstyle=[1]\textbf,    % Stil der Keywords
 %        keywordstyle=[2]\textbf,    %
 %        keywordstyle=[3]\textbf,    %
 %        keywordstyle=[4]\textbf,   \sqrt{\sqrt{}} %	
         stringstyle=\color{white}\ttfamily, % Farbe der String
         showspaces=false,           % Leerzeichen anzeigen
         showtabs=false,             % Tabs anzeigen ?
         xleftmargin=17pt,
         framexleftmargin=17pt,
         framexrightmargin=5pt,
         framexbottommargin=4pt,
         %backgroundcolor=\color{lightgray},
         showstringspaces=false,      % Leerzeichen in Strings anzeigen ?
         captionpos=b,
		language=C++,
 }

\begin{document}
 
\begin{abstract}

Photon mapping is a popular extension to the classic ray tracing algorithm in the field of realistic image synthesis. Moreover  it is benefit from the massive parallelism computation power brought by the recent development of graphics processor hardware and programming model. However rendering the scenes with dynamic lights still greatly limits the performance due to the re-construction of kd-tree for the photons. We developed a novel approach based on the idea that storing the photons data with kd-tree leaf nodes data and implemented new incremental update scheme to improve the performance for dynamic lighting. The implementation is GPU-based and fully paralleled. A series of benchmarks with existing GPU photon mapping technique is carried our to evaluate our approach and our new technique is shown to be faster when handling the dynamic lighted scenes than the existing technique while having the same image quality. 

\end{abstract}

% TF: Need title page & abstract.

\begin{acknowledgments}
First of all, I would like to thank my supervisor, Dr.~Thomas Fevens for his advice, encouragement and support throughout my research. I would also like to thank my parents and my girlfriend Chunxiao Ma who are supportive as always.
\end{acknowledgments}

%%%%%%%%%%%%%%%%%%%%%%%%%%%%%%%%%%%%%%%%%%%%%%%%%%%%%%%%%%%%%%%%%%%%%%%%%%%%%%%
%% Body of Thesis goes here.
%%%%%%%%%%%%%%%%%%%%%%%%%%%%%%%%%%%%%%%%%%%%%%%%%%%%%%%%%%%%%%%%%%%%%%%%%%%%%%%

% Set the page numbering style
\pagenumbering{arabic}	
\setcounter{page}{1}

%% double spacing
\begin{spacing}{2.0}
	
%-----------------------
% Chapter 1 : Introduction
\chapter{Introduction}

Due to the recent development of Graphics Process Units (GPUs) hardware architecture and the massive parallel programming model that enables developers to fully exploit the computation power of GPUs, parallel computing using GPU has become more and more popular to develop high performance applications. Realistic image synthesis, especially global illumination which is one of the most computationally complex algorithm, has drawn much research interests focusing on developing new parallelized approaches to unleash the computation power of GPU. Photon mapping is one of the most popular global illumination approaches these days. Many acceleration techniques developed for photon mapping can achieve very good performance for static scenes. However, dynamic scenes such as the scene in which moving lights sources may have a big impact on the performance of renderer. One of biggest causes of this overhead is that the the dynamic scene requires the reconstruction of the acceleration data structure every frame. The data structure for photon mapping is usually balanced kd-tree \cite{Bentley:1975:MBS:361002.361007} used to speed up the photon search. The parallelize kd-tree construction algorithm will greatly increase the peak memory consumption though it is a pretty fast algorithm. 

In this thesis we would like to present a novel approach based on the traditional photon mapping technique on GPU.  Using this approach we could rearrange the photons data to avoid the reconstruction of kd-tree for dynamic scenes with moving light sources, an incremental update algorithm is also employed so that we can achieve the same image quality with better performance. In order to proof our proposed technique in terms of speed, memory consumption and image quality, we will implement the prototype of the existing technique and compare it with the new approach. 

\section{Structure of the thesis} 

Following the introduction, we will firstly introduce the most important theoretical concepts and frameworks to establish a foundation for the subsequent discussion of global illumination approaches. Then we will make a short survey on various approaches and their strength and weakness when dealing with different phenomena. Afterwards we will highlight GPU-based techniques applied on Monte Carlo ray tracing and photon mapping and discuss some open issues of the current approach for the introduction of our new approach in the following chapter.  In chapter three, we presented a new GPU photon mapping approach with detailed information on our current implementation. The results of our experiment and the analysis of the results are presented in chapter four. Finally, in chapter five we come to the conclusion, and discuss the directions for future work.

%-------------------------
% Chapter 2:  Background

% Theoretical
%- Radiometry Introduction.
%- BRDF 	
%- Rendering equation and LTE equation
%- Monte-Carlo Ray Tracing.
%- Photon Mapping

% Technical
%- KD-Tree Construction and representation on GPU
%- Photons KNN search using KD-Tree
\chapter{Background}

To establish the mathematics basis and mental model of the problem, some essential conceptions in Computer Graphics are going to be introduced in this chapter.  

%%%%%%%%%%%%%%%%%%%%%%%%%%%%%%%%%%%%%%%%%%%%%%%%%%%%%%%%%%%%%%%%%%%%%%%%%%

\section{Radiometry Introduction}
Radiometry is the basic terminology to describe light which is crucial to simulation. First of all, some basic quantities have to be introduced, the related symbols are going to be defined here as well for further use.

\subsection{Important Quantities} 

\begin{table}[!ht]
\begin{center}
	\begin{tabular}{ | l | l | l |}     	
	\hline 

	Symbol & Quantity & Unit \\

	% \(Q_{\lambda}\) 	& 		Spectral radiant energy 		& 		\(J nm^{-1} \) \\
	\(Q\) 			& 		Radiant Energy 				& 		\(j\) \\ 
	\(\Phi\) 			& 		Radiant flux 					& 		\(W\) \\ 
	\(I\) 			& 		Radiant intensity 				& 		\(W sr^{-1}\) \\
	\(E\)			&		Irradiance (incident) 			&		\(W m^{-2}\) \\  
	\(L\)			&		Radiance						&		\(W m^{-2} sr^{-1}\) \\ 
	
	\hline

	\end{tabular}
\end{center} 
\caption{Radiometric symbols, names and units.}
\label{tab:radiometry_quantities}
\end{table}

\emph{Radiant energy}, \(Q\), is the energy of a collection of photons which is the basic quantity in lighting. 

\emph{Radiant flux} , \(\Phi\), is the time rate of the flow of radiant energy passing through a surface or region of space. Total emission from a light source is generally described in terms of flux. \ref{fig:flux_point_light} shows the flux emitted from a point light source measured by the total amount of energy passing through an virtual sphere around the light. 

\begin{figure}[htp] 
    \centering 
    \fbox{\includegraphics{flux.pdf}}
    \renewcommand{\thefigure}{\thechapter.\arabic{figure}}
    \caption[]{Radiant flux from a point light source is passing through the spheres around the light.}
    \label{fig:flux_point_light} 
\end{figure} 

\emph{Irradiance}, \(E\), is the incident (arriving at a surface location) \emph{radiant flux area density}, which is defined as the differential flux per differential area. While \emph{Radiant exitance} denoted by \(M\) is area density of flux leaving a surface.  

\begin{equation}
E(x) = \frac{d\Phi}{dA}
\end{equation}

\emph{Radiance}, \(L\), is the radiant flux per unit solid angle per unit projected area: 

\begin{equation}
L(x, \overrightarrow{\omega}) = \frac{d^{2}\Phi}{\cos{\theta} \cdot dA \cdot d\overrightarrow{\omega}}
\end{equation}

Where \(x\) is the position and \(\overrightarrow{\omega}\) is the direction. 

Radiance is the most important quantity in rendering simulation since it closely represent the color. Also radiance can be considered as the number of photons arriving per time at a small area from a given direction. Radiant energy can be computed by integrating the radiance field over all directions \(\Omega\) and area \(A\).

\begin{equation} 
\Phi = \int_{A}\int_{\Omega}L(x, \overrightarrow{\omega})(\overrightarrow{\omega} \cdot \overrightarrow{n})d\overrightarrow{\omega}dx
\label{eq:flux_from_radiance}
\end{equation} 

\begin{figure}[htp] 
    \centering 
    \fbox{\includegraphics{solid_angle_sphere.pdf}}
    \renewcommand{\thefigure}{\thechapter.\arabic{figure}}
    \caption[]{Radiance, L, is defined as the radiant flux per unit solid angle, \(\overrightarrow{\omega}\), per unit projected area, \(dA\)}
    \label{fig:solid_angle_sphere} 
\end{figure}

The solid angle used in equation \ref{eq:flux_from_radiance} can be thought as representation of both a direction and an infinitesimal area. Therefore solid angle can also be expressed in spherical coordinates (\(\theta, \phi\))

\begin{figure}[htp] 
    \centering 
    \fbox{\includegraphics{radiance_solid_angle.pdf}}
    \renewcommand{\thefigure}{\thechapter.\arabic{figure}}
    \caption[]{Radiance, L, is defined as the radiant flux per unit solid angle, \(\overrightarrow{\omega}\), per unit projected area, \(dA\)}
    \label{fig:radiance_solid_angle} 
\end{figure} 

\subsection{Light Source}
The light in the form of photons is emitted from light sources. We can measure the intensity of light source in \emph{wattage}. Take a point light source for example, the power this light can emit is denoted by \(\Phi\), the emitted light distribute uniformly in all directions, the irradiance, \(E\), can be computed at a surface as: 

\begin{equation}
E(x) = \frac{\Phi \cos{\theta}}{4\pi r^{2}} 
\end{equation}

Where \(r\) is the distance from \(x\) to the light source and \(\theta\) is the angle between the surface normal and the direction to the light source. From the equation we can intuitively tell a surface facing the source will receive more photons per area than a surface that is oriented differently.   

%%%%%%%%%%%%%%%%%%%%%%%%%%%%%%%%%%%%%%%%%%%%%%%%%%%%%%%%%%%%%%%%%%%%%%%%%%


\section{Render Equation}
With the physically-based model of lighting, a theoretical framework describing the interaction between light and an surface will be introduced in this section. 

\begin{figure}[htp] 
    \centering 
    \fbox{\includegraphics{brdf.pdf}}
    \renewcommand{\thefigure}{\thechapter.\arabic{figure}}
    \caption[]{The geometric setup of BRDF. }
    \label{fig:brdf} 
\end{figure} 

The \emph{Bidirectional Reflectance Distribution Function}, BRDF, is the mathematic tool describing the reflectiona of light encounters an surface. To define the BRDF, the geometric configuration is shown in figure \ref{fig:brdf}. \(\omega_{i}\) is the incident lighting direction, \(\omega_{o}\) is the direction in which the reflected light leaving from the surface, \(n\) is the normal vector at the location \(p\) on the surface. Given the incident radiance \(L_{i}(p, \omega_{i})\), we are finding out the outgoing radiance to the viewer, \(L_{o}(p, \omega_{o})\). 

The BRDF, \(f_{r}\), defines the relationship between differential reflected radiance and differential irradiance: 

\begin{equation}
f_{r} = \frac{dL_{o}(p, \omega_{o})}{dE(p, \omega_{i})} = \frac{dL_{o}(p, \omega_{o})}{L_{i}(p, \omega_{i})(\omega_{i} \cdot n)d\omega{i}}
\label{eq:brdf}
\end{equation} 

There are two important properties of BRDF used in rendering. The first one is the Helmholtz's law of reciprocity, that is given any pair of directions \( (\omega_{i}, \omega_{o} ) \), we have: 

\begin{equation}
f_{r}(p, \omega_{i}, \omega_{o}) = f_r(p, \omega_{o}, \omega_{i})
\end{equation}

Another important physical property of BRDF is energy conservation, stating that the total reflected energy is less than or equal to the incident energy. For all direction \( \omega_{o} \).

\begin{equation}
 \int_{\Omega}f_{r}(p, \omega_{i}, \omega_{o})L_{i}(p, \omega_{i})(\omega_{i} \cdot n)d\omega_{i} \leq 1 , \forall \omega_{i}
\end{equation}

Given the definition of BRDF, we can introduce the basic render equation, also known as \emph{local illumination model} by integrating the equation \ref{eq:brdf} over the sphere of incident directions around location \(p\), the left side of the equation will be the outgoding radiance in direction \(\omega_{o}\).

\begin{equation}
L_{o}(p, \omega_{o}) = \int_{\Omega}f_{r}(p, \omega_{i}, \omega_{o})L_{i}(p, \omega_{i})(\omega_{i} \cdot n)d\omega_{i}
\label{eq:local_render_equation}
\end{equation}

Where \(\Omega\) is the hemisphere of incoming directions at \(p\).

\paragraph{Light Transport Equationa} 

The local illumiation model is used to describe the direct lighting effect which is too simple for simulating real-world lighting effect, indirect lighting has to be introduced to this model as well. Therefore we introduce the Light Transport Equation (LTE) in this section to form the mathematical basis for all global illumination algorithms, The LTE states the necessary conditions for equilibrium of light transport in the scene without participating media. 

\begin{equation}
L_{o}(p, \omega_{o}) = L_{e}(p, \omega_{o}) + \int_{\Omega}f_{r}(p, \omega_{i}, \omega_{o})L_{i}(p, \omega_{i})(\omega_{i} \cdot n)d\omega_{i}
\label{eq:lte}
\end{equation}

%%%%%%%%%%%%%%%%%%%%%%%%%%%%%%%%%%%%%%%%%%%%%%%%%%%%%%%%%%%%%%%%%%%%%%%%%%

\section{Monte-Carlo Ray Tracing} 




%-------------------------
% Chapter 3: New Approaches
\chapter{A New Approach}

In this chapter, we will present a noval approach for rendering a scene with dynamic light source efficiently. This method is based d on the standard GPU-based photon mapping rendering system, using an augmented kd-tree data structure and localized updating algorithm for rendering.  

In the first 2 sections of this chapter, We will have an overview on our approach and comprehensive description of the data structure that we use and related algorithms. Then we will look at some of the GPU implementation details. 

\section{Overview} 

As described in chapter 2, we build a gloabl kd-tree as the photon map for irradiance estimation. However, the global photon map is rebuilt from scratch every frame, this process is time consumings due to the complexity of kd-tree building algorigthm on GPUs and can be avoided. 

Instead of building a global kd-tree for photons, we re-use the same kd-tree for the geometry objects(triangles) which is static and associate the geometry and the photons data for efficient KNN search and update for dynamic scene. 

For each frame, the photons are shot from the light source and stored in an array, then we build the kd-tree for the geometry objects if it is required, for the scenes that don't include animated objects we will skip this process. Given the photons data in the scene and the built kd-tree for geometry, we build our data structure, compress the memory bind it to the texture memory for a better memory accessing performance. In rendering phase, instead of using photon map for radiance estimation, we use the kd-tree and our photon queue perform KNN search. Further details of the data structure will be presented in the following section. 

The update process of photons queue is straightforward. It depends on the photon data from the previouse frames. Here we keep track of all the photons data of a range of frames with a pair of indices, the indices can be updated and maintained efficiently. Furthur  details will be presented in following section.  

\section{Data Structure And Algorithm} 

\subsection{Data Structure}

\begin{figure}[htp] 
    \centering 
    \fbox{\includegraphics{kd_leaf_photons.pdf}}
    \renewcommand{\thefigure}{\thechapter.\arabic{figure}}
    \caption[]{Radiant flux from a point light source is passing through the spheres around the light.}
    \label{fig:kd_leaf_photons} 
\end{figure}  

As shown in figure \ref{fig:kd_leaf_photons}, in addition to the triangle data, the photons data in the scene is also logically associated with the leaf nodes of the kd-tree. As the kd-tree for the scene actually encodes the spatial relationship among the geometry, and the photons will be stored when they hit the objects with diffuse objects, we attach the photons fall into the spatial region occupied by the bouding box of a kd-tree leaf node. 

\begin{figure}[htp] 
    \centering 
    \fbox{\includegraphics[width=\linewidth]{kd_leaf_photons_2.pdf}}
    \renewcommand{\thefigure}{\thechapter.\arabic{figure}}
    \caption[]{Radiant flux from a point light source is passing through the spheres around the light.}
    \label{fig:kd_leaf_photons_2} 
\end{figure}  

In figure \ref{fig:kd_leaf_photons_2} we have a more detailed view on how the photons data organized. The photons shot to the scene in one frame of rendering are stored followed by the photons of next frame. When implementing this organization on GPU, all actual photons data (positions, incident directions and power) is stored in a seperate array, the indices of the photons associated with kd-tree leaf nodes are stored instead of the actual photon data. 

\subsection{Construction} 

Building the logical connection between photons and leaf nodes is simpler than build a global kd-tree for the photon map. Firstly, we found out the the leaf nodes from the built kd-tree in parallel. This can be done checking a flag set when building the kd-tree. Given the indices of leaf nodes, we can retrieve the bounding box from the small node list generated in kd-tree construction phase(see \ref{subsec:kdtree_construction}). Then we check which leaf node certain group of photons should fall into in parallel by performaing fast point-box intersection testing. For each leaf node, the number of photons can be calculated with a parallel reduction operation. Further details on physical memory arrangement and implementation of the data structure  will be presented in section \ref{sec:impl_detials}. 

\subsection{Update} 

When a frame of image is rendered, new photons will be shot into the scene and be stored in the same global array with previous frames and for each leaf node, we need to keep track of the range of the photons that are active for rendering the next frame. For each leaf node, we maintain two pointers(indices), start and end index, indicating the range of the photons used for rendering. We move the end pointer forward as there are new photons comming, move the start pointer forward when there are some old photons we need to discard. We use a pre-defined threshold fram count to determine how manay frames of photons data we want to make active, we keep accumulating the photons every frame until we reach that threshold value. When the threshold frame count 
is reached, we move forward the start pointer to avoid there are too many photons than there should be. Since the number of photons per frame is known, the stride of moving start pointer can be calculated. However, in GPU implementation, the step in bytes used to increment the pointer is usually larger than the exact size of photons data we want to discard, since the memory will be padded when it is allocated for better memory accessing performance. Further details will be presented in section \ref{sec:impl_detials}. 

\subsection{Rendering} 

Rendering with our data structure does not differ from the standard photon mapping rendering algorithm. In rendering phase, we cast the rays from the camera into the scene in parallel and perform KNN search using the kd-tree already built for the geometry, when we reach the leaf nodes we use our data structure to find the photons data for particular leaf node and gather the photons for radiance estimation. 

\section{Implementation Details} 
\label{sec:impl_detials} 

\subsection{Data Organization}

\paragraph{KD-Tree Data} 

The kd-tree data should be carefully organized when implemented with CUDA to improve traversal performance. In \ref{subsec:kdtree_construction}, we have already described the kd-tree building algorithm introduced in \cite{Zhou2008}, but some of the details that are critical to the program's overall performance still need to be discussed here. 

The final node list generated when the kd-tree is built is not sufficient for fast traversal algorithm, it contains too much useless information and will hit the performance since it is not friendly to cache prefetch. Therefore we compress and re-organize the traversal related data of the kd-tree nodes to reduce the memory access. The entire kd-tree is stored in a structure of several contiguous arrays. The structure is defined as following:  

\lstinputlisting{kdtree_data_def.cpp} 

of unsigned integer and is organized in pre-order, as shown in the following figure: 

\begin{figure}[htp] 
    \centering 
    \fbox{\includegraphics[width=\linewidth]{kdtree_data_memory_layout.pdf}}
    \renewcommand{\thefigure}{\thechapter.\arabic{figure}}
    \caption[]{Compacted kd-tree data memory layout.}
    \label{fig:kdtree_data_memory_layout} 
\end{figure}  

The whole left subtree is stored before right child and its subtree due to the pre-order traversal, this memory layout improves the cache performance. 


\paragraph{Photons Data} 


\subsection{GPU memory management}


\paragraph{GPU Texture Memory VS Global Memory} 


\paragraph{Shared Memory} 








%-------------------------
% Chapter 4: Result and Analysis
\chapter{Result and Analysis}


\section{Timing}

\subsection{KD-Tree Construction on GPU} 

\subsection{Photons Queue Construction}

\section{Memory Consumption} 



\section{Image Quality}



	

%o-------------------------
% Chapter 5: Conclusion
\chapter{Conclusions}

%\section{Conclusions}
In this research, we mainly focus on developing an improved photon mapping technique specifically for current generation GPU taking the advantage of the massive parallelized computation power. We started with introducing the theoretic basis of global illumination, then we reviewed the existing approaches published on the solving the problem, especially the photon mapping techniques exploited the parallelism with GPU. In our opinion, the existing method suffers from a deficiencies in how it handles the dynamic scene. After analyzing the weakness, we present a new approach trying to avoid another kd-tree construction exclusively for photon data and support an incremental updating scheme by associating the photons data with the kd-tree of the geometric objects and accumulating the photons from previous frames for K Nearest Neighbor photon searching. To proof our concept we implemented the test programs for both the existing and our new approach and carried out a series of benchmarks.

% TF: Add a brief point-form summary of what is new/different about your approach.
% Steve: done
The major contribution of this thesis can be summerized as follows: 

\begin{itemize}
 
  \item Proposed a new data structure that maintians the kd-tree leaf nodes data and photons data and implemented the contruction and incremental updating algorithm on this data structure allowing us to efficiently perform radiance estimation using photons. 
  \item Improved the rendering performance for the scenes with dynamic light source by using the new data structure to avoid the overhead brought by re-constructing another kd-tree for photons every frame. 
  \item Achieved the same image quality compares to the standard photon mapping technique. 
 
\end{itemize}

In our tests we observed that our approach was faster during construction and almost all the photon search tests among certain range of frames for a dynamic scene. Especially the construction time was greatly reduced compared to the old approach. Along with speed measurements, we also examined the memory consumption of both approaches. Our approach consumes more memory in rendering phase since the photon data from previous frames. But the memory consumption in the construction phase is much less than the existing method. Because we only require the construction of kd-tree for the scene. The memory consumption also strongly depends on the number of frames we want to store for the photons. Finally, we validate the quality of the rendered by directly visualizing the photons in the scene.

\section{Future Work}

We believe that there are a couple of interesting topics the future work can work on to improve our new approach. The first one is improve the algorithm of the classifying the photons to the kd-tree's leaf nodes to achieve more efficient parallelization. The current solution is that each thread works on one leaf node iterating over all the photons, this could lead considerable deficiency of parallelization on GPU especially when the height of the tree is relative low (there are less leaf nodes) due to the low occupancy of CUDA threads. One possible solution is to map the photons to CUDA threads instead, marking the photons that belong  to certain kd-tree leaf node and applying a parallel primitive algorithm such as compact to calculate the indices of photons for each kd-tree leaf node in parallel.

Another interesting application we can explore further is to test and observe that how participating media such as smoke, dust or clouds effect the performance using our approach. The participating media will affect light when it travels through them, the light beam is either absorbed or scattered. Since our approach spatially encodes the distribution of the photons based on the volume boundary of the kd-tree nodes, we think our approach is suitable for storing photon information of volumetric participating media and good performance could be expected.

In addition to keep evolving the approach algorithmically, we certainly cannot ignore the impact brought by the developments in graphics hardware on the photon mapping techniques. With the latest generation of GPUs and development framework, the applications will benefit from better GPU performance and more flexibility, such as better performance for non-coalesced memory access implicit hardware optimizations which is almost free for developers. 		

%-------------------------
% Appendix
%\chapter{Appendix}

\section{Introduction to CUDA Parallel Programming}
 				
%-------------------------

\bibliographystyle{plainnat} 
\bibliography{refs}

\end{spacing}


\end{document}
